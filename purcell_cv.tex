% LaTeX Curriculum Vitae Template
%
% Copyright (C) 2004-2008 Jason Blevins <jrblevin@sdf.lonestar.org>
% http://jblevins.org/projects/cv-template
%
% You may use use this document as a template to create your own CV
% and you may redistribute the source code freely. No attribution is
% required in any resulting documents. I do ask that you please leave
% this notice and the above URL in the source code if you choose to
% redistribute this file.

\documentclass[letterpaper]{article}

\usepackage{hyperref}
\usepackage{geometry}

% Uncomment the following lines to use the Palatino font.  Remove the
% [osf] bit if you don't like the old style figures.
%
% \usepackage[T1]{fontenc}
% \usepackage[osf]{mathpazo}

% Set your name here
\def\name{Kevin M. Purcell}

% The following metadata will show up in the PDF properties
\hypersetup{
  colorlinks = true,
  urlcolor = black,
  pdfauthor = {\name},
  pdfkeywords = {},
  pdftitle = {\name: Curriculum Vitae},
  pdfsubject = {Curriculum Vitae},
  pdfpagemode = UseNone
}

\geometry{textheight=8.5in, textwidth=6.5in}

% Customize page headers
\pagestyle{myheadings}
\markright{\name}
\thispagestyle{empty}

% Customize section headings
\usepackage{sectsty}
\subsectionfont{\rmfamily\mdseries\itshape\large}

% Don't indent paragraphs.
\setlength\parindent{0em}

% Make lists without bullets
\renewenvironment{itemize}{
  \begin{list}{}{
    \setlength{\leftmargin}{1em}
  }
}{
  \end{list}
}

\begin{document}

\centerline{\huge\bf \name}
\hrulefill
\vspace{0.25in}

\begin{minipage}[t]{0.5\textwidth}
  \href{http://nmfs.noaa.gov/}{National Marine Fisheries Service}\\
  \href{http://www.sefsc.noaa.gov/labs/beaufort/}{Southeast Fishery Science Center}\\
  \href{http://www.biology.ull.edu/}{Beaufort Laboratory} \\
  101 Pivers Island Road \\
  Beaufort, NC 28516-9722 \\
  %Office: (252) 728-8761 \\
  %Fax: (252) 728-8619

\end{minipage}
\begin{minipage}[t]{0.5\textwidth}
  Office: (252) 728-8761 \\
  Fax: (252) 728-8619 \\
  Email: \href{mailto:kevin@kevin-purcell.com}{\tt kevin@kevin-purcell.com} \\
  Homepage: \href{http:www.kevin-purcell.com/}{\tt www.kevin-purcell.com} \\
  ORCiD: \href{http://orcid.org/0000-0001-6046-2774}{0000-0001-6046-2774} \\
  %twitter: \href{https://twitter.com/kevin_purcell}{\tt @kevin\_purcell} \\
  %GitHub: \href{https://github.com/kpurcell}{\tt kpurcell}
\end{minipage}


\section*{Education}
	\begin{itemize}
		\item \textbf{Department of Biology, University of Louisiana at Lafayette}\\
    Ph.D., Environmental and Evolutionary Biology, 2009. Advisor: Dr. Paul L. Leberg

		\item \textbf{Department of Biological Sciences, Bloomsburg University} \\
			M.S. Biology, 2002. Advisor: Dr. Thomas Klinger
      
		\item \textbf{Biological Sciences Deppartment, East Stroudsburg University} \\
			B.S. Biology and B.S. Marine Science, 1999.
	\end{itemize}


\section*{Professional Experience}
  \begin{itemize}
			\item \textbf{Southeast Fishery science Center, National Marine Fishery Service}\\
      Postdoctoral Associate, 2011--Present

			\item \textbf{Environmental and Conservation Program, North Dakota State University} \\
			ECS Postdoctoral Fellow, 2009--2011

			\item \textbf{Neigel Lab, University of Louisiana at Lafayette}  \\
       Research Assistant, 2008--2009 
      
			\item \textbf{Department of Biology, University of Louisiana at Lafayette} \\
       Instructor, 2006--2008 
      
			\item \textbf{Department of Biology, University of Louisiana at Lafayette} \\
      University Fellow, 2003--2006
      
	\end{itemize}

 
\section*{Peer-Reviewed Publications}
	\begin{itemize}
    \item M. Karnauskas, M.J. Schirripa, J.K. Craig, G. Cook, C. Kelbe, J. Agar, B. Black, D. Enfield, D. Lindo-Atichati, B. Muhling, \textbf{K.M. Purcell}, P. Richards, C. Wang.  2014. Evidence of climate-driven ecosystem reorganization in the Gulf of Mexico.  \textit{Proceeding of the National Academy of Science}. (\textit{In review})
		\item B.J. Langseth, \textbf{K.M. Purcell}, J.K. Craig, A.M. Schueller, J.W. Smith, K. W. Shertzer, S. Creekmore, K.A. Rose, K. Fennel. 2014. Effect of changes in dissolved oxygen concentrations on the spatial dynamics of the Gulf Menhaden fishery in the northern Gulf of Mexico. \textit{Marine and Coastal Fisheries}. (\textit{In review})
		\item \textbf{K. M. Purcell} and C. A. Stockwell. 2013. Mechanism dictating the structure and dispersal of a non-native invasive fish species to New Zealand. \textit{Biological Invasions}. (\textit{In review})
		\item D. Enfield, G. Goni, F. Bringas, D. Lindo, S.-Ki Lee, Y. Liu, B. Muhling, L. Avens, J. Agar, J. Litz, P. Richards, L. Garrison, \textbf{K. M. Purcell}, J. Smith, N. Bartlein, T. Minello, B. McAnally, D. Apeti, H. Perryman. (editors) M. Kamauskas, M. J. Schirripa, C. Kelble, G. Cook, J. K. Craig. 2013. Gulf of Mexico Ecosystem Status Report. \textit{NOAA Technical Memorandum} #NMFS-SEFSC-653.
		\item C. A. Stockwell, J. S. Heilveil, \textbf{K. M. Purcell}. 2012. Estimating divergence time for two evolutionarily significant units of a protected fish species. \textit{Conservation Genetics} 14:215-222.
		\item \textbf{K. M. Purcell}, P. L. Klerks, A. T. Hitch, P. L. Leberg. 2012. The role of genetic structure in the adaptive divergence of populations experiencing saltwater intrusion due to relative sea-level rise. \textit{J. Evolutionary Biology} 25:2623-2632.
		\item \textbf{K. M. Purcell}, N. Ling, C. A. Stockwell. 2012. Evaluation of the introduction history and genetic diversity of a serially introduced fish population in New Zealand. \textit{Biological Invasions} 14:2057-2065.
		\item C. A. Stockwell, \textbf{K. M. Purcell}, M. L. Collyer, J. Janovy. 2011. Translocations alter parasite communities of a protected species due to the environmental tolerance mismatch among hosts. \textit{Transactions of the American Fisheries Society} 140:1370-1374.
		\item \textbf{K. M. Purcell}, S. L. Lance, K. L. Jones, C.A. Stockwell. 2011. Ten novel microsatellite markers for the western Mosquitofish, \textit{Gambusia affinis}. \textit{Conservation Genetic Resources} 3: 361-363.
		\item A. T. Hitch, S. B. Martin, \textbf{K. M. Purcell}, P. L. Klerks, P. L. Leberg. 2010. Interactions of salinity, marsh fragmentation and submerged aquatic vegetation on resident nekton assemblages of coastal marsh ponds. \textit{Estuaries and Coasts} 34: 653-662.
		\item \textbf{K. M. Purcell}, P. L. Klerks, P. L. Leberg. 2010. Adaptation to sea level rise: Does local adaptation influence the demography of coastal fish populations. \textit{Journal of Fish Biology} 77:1209-1218.
		\item S. B. Martin, A. T. Hitch, \textbf{K. M. Purcell}, P. L. Klerks, P. L. Leberg. 2009. Life history variation along a salinity gradient in coastal marshes. \textit{Aquatic Biology} 8:15-28.
		\item \textbf{K. M. Purcell}, A. T. Hitch, P. L. Klerks, P. L. Leberg. 2008. Adaptation as a potential response to sea-level rise: a genetic basis for salinity tolerance in populations of a coastal marsh fish. \textit{Evolutionary Applications} 1: 155–160.
		\item P. L. Leberg, M. C. Green, B. A. Adams, \textbf{K. M. Purcell}, M. C. Luent. 2007. Response of waterbird colonies in southern Louisiana to recent drought and hurricanes. \textit{Animal Conservation} 10: 502–508.
	 \end{itemize}
 
\subsection*{Publications in development}
	\begin{itemize}
		 \item \textbf{K. M. Purcell}, J. Nance, M. Smith, J. K. Craig. 2014. Coastal hypoxia and its effects on the spatial dynamics of a large marine fishery. (44 pgs, 9 fig, 1 table)
		 \item \textbf{K. M. Purcell} and J. K. Craig. 2014. Global fishery imports and Eutrophication: synergistic effects on the demersal nekton community of the Gulf of Mexico. (37 pgs, 5 figs, 2 tables)
	\end{itemize}

\section*{Grants}
	\begin{itemize}
		 \item \textbf{K. M. Purcell}, J.K. Craig, K. Andrews, J. Nance, E. Scott-Denton. 2014. A spatial analysis of bycatch estimation methodologies for the Gulf of Mexico shrimp fishery. \textit{Information to Support and Conduct Stock Assessments Program}. (\textbf{\$97,000}) (\textit{In review})
		 \item \textbf{K.M. Purcell}, J.K. Craig, J.W. Smith.  2014.  Unmanned Aircraft Systems (UAS) as a tool for fishery-independent data collection in support of fisheries management.  \textit{NOAA Fisheries Advanced Sampling Technology Working Group}. (\textbf{\$122,000}) (\textit{In review})
		 \item \textbf{K. M. Purcell}, J.K. Craig, K. Andrews, J. Nance, E. Scott-Denton, J. Primrose. 2014. Spatial dynamics of bycatch in the Gulf of Mexico shrimp fishery. \textit{Marine Fisheries Initiative}(MARFIN). (\textbf{\$95,000}) (\textit{In review})
		 \item S. T. Walter, J. Karubian, \textbf{K. M. Purcell}, J. K. Craig, P.L. Leberg. 2012. Effects of hypoxia on brown pelican foraging ecology and demographic processes. \textit{National Geographic}. (\textbf{\$20,000})
		 \item J. D. Fisher, \textbf{K. M. Purcell}, C. A. Stockwell. 2009. A genetic evaluation of northern leopard frog populations in North Dakota. \textit{North Dakota State Wildlife Grant}. (\textbf{\$100,000})
		 \item \textbf{K.M. Purcell}. 2009. The Kiwi Invasion: the structure and diversity of a serially introduced fish species to the coastal environments of northern New Zealand. \textit{Conservation Genetics Research Fellowship Award, North Dakota State University}. (\textbf{\$20,000})
     
		 \item \textbf{K. M. Purcell}, P. L. Leberg. 2007. Evaluations of spatial variability in genetic adaptations to saltwater intrusion. \textit{Coastal Restoration & Enhancement through Science & Technology Grant}. (\textbf{\$2,500})
	\end{itemize}

\section*{Teaching}

  \subsection*{North Dakota State University}
    \begin{itemize}
      \item Conservation Biology (undergraduate level, 2010, 2011)
      \item Conservation Genetics (graduate level, 2010, 2011)\\
    \end{itemize}
    
  \subsection*{University of Louisiana at Lafayette}
    \begin{itemize}
      \item Fundamentals of Biology I (undergraduate level, 2008)
      \item Biological Principles \& Issues I (undergraduate level, 2006, 2007)
      \item Biological Principles \& Issues II (undergraduate level, 2007)
      \item Biological Principles \& Issues I Lab (undergraduate level, 2006)
      \item Marine Invertebrate Zoology Lab (undergraduate/graduate, 2006)
    \end{itemize}
   
   
\section*{Published abstracts and Presentations}
	\begin{itemize}
		\item \textbf{K. M. Purcell}, J. K. Craig, M. D. Smith, J. M. Nance. 2013. The effects of hypoxia on the spatial dynamics of the northwestern Gulf of Mexico. \textit{Aquatic Sciences meeting of the American Society for Limnology and Oceanography}. New Orleans, LA.
		\item J. D. Fisher, \textbf{K. M. Purcell}, C. Stockwell. 2013. Landscape influences on effective population size of Northern Leopard Frog in North Dakota. \textit{50th North Dakota Chapter of the Wildlife Society}. Mandan, ND.
		\item \textbf{K. M. Purcell} and C. Stockwell. 2012. Evaluation of the introduction history and genetic diversity of a serially introduced fish population in New Zealand. 1st North America Congress for Conservation Biology. Oakland, CA.
		\item J. Fisher, \textbf{K. M. Purcell}, C. Stockwell. 2012. Genetic Diversity of the Northern Leopard frog across the 100th meridian. 1st North America Congress for Conservation Biology, Oakland, CA. (poster)
		\item C. Stockwell, J. Heilveil, \textbf{K. M. Purcell}. 2012. Estimating time of divergence for two evolutionary significant units of a protected fish species. 1st North American Congress for Conservation Biology. Oakland, CA.
		\item \textbf{K. M. Purcell} and J. K. Craig. 2012. The Impact of Hypoxia and Spatial Distribution on Catchability in the Gulf of Mexico Shrimp Fishery. 26th Annual Meeting of the American Fisheries Society, Tidewater Chapter, Beaufort, NC.
		\item \textbf{K. M. Purcell}, N. Ling, C. Stockwell. 2011. Evaluation of the introduction history and genetic diversity of serially introduced mosquitofish populations in New Zealand. 25th International Congress for Conservation Biology. Auckland, NZ.
		\item \textbf{K. M. Purcell} and C. A. Stockwell. 2011. Evaluation of the introduction history and genetic diversity of a serially introduced fish population in New Zealand. Joint Meeting of Icthyologists and Herpetologists, Minneapolis, MN.
		\item \textbf{K. M. Purcell} and P. L. Leberg. 2010. The impact of sea level rise and phenotypic divergence on the genetic structure of a coastal marsh fish. Evolution, Portland, OR.
		\item \textbf{K. M. Purcell} and P. L. Leberg. 2008. Adaptation to salinity stress: the role of genetic architecture and geographical separation. Ecological Society of America’s 93rd Annual International Meeting. Milwaukee, WI.
		\item \textbf{K. M. Purcell} and P. L. Leberg. 2008. Spatial Differences in genetic adaptation to saltwater intrusion. Coastal Restoration and Enhancement through Science and Technology’s Annual Award Symposium. New Orleans, LA.
		\item \textbf{K. M. Purcell} and P.L. Leberg. 2005. Effects of historical salinity exposure on coastal mars fish populations. Estuarine Research Federation’s 18th Biennial International Conference. Norfolk, VA.
		\item \textbf{K. M. Purcell} and PL Leberg. 2004. Salinity Resistance in coastal marsh fish populations. Gulf Estuarine Research Society. Pensacola, FL.
	\end{itemize}

\section*{Invited Seminars}
	\begin{itemize}
		 \item \textbf{K. M. Purcell}, M. Smith, L. Bennear, J. Nance, , J. K. Craig. 2013. Non-linear modeling of coastal hypoxia and its impacts on the Gulf of Mexico shrimp fishery. \textit{Forum for Gulf of Mexico Hypoxia Research Coordination and Advancement}, Mississippi State University, Stennis Space Center, MS.
		 \item \textbf{K. M. Purcell}, J. Nance, J. K. Craig. 2012. Ecological and economic effects of hypoxia on the gulf shrimp fishery. \textit{3rd Annual Hypoxia Coordination Workshop}, Bay St. Louis, MS.
		 \item \textbf{K. M. Purcell}. 2011. The eco-evolutionary dynamics of salinity stress on a coastal nekton population. \textit{National Oceanic and Atmospheric Association}, Beaufort, NC.
		 \item \textbf{K. M. Purcell}, P. L. Leberg. 2010. Adaptation to salinity stress; the response of coastal fish populations to a changing world. \textit{Department of Biology, Rivier College}, Nashua, NH.
		 \item \textbf{K. M. Purcell}. 2009. Implications of changing climate and sea level rise on coastal communities. \textit{4th Annual Global Climate Teach-In}, Fargo, ND.
		 \item \textbf{K. M. Purcell}. 2008. The genetics of adaptation to sea level rise: evolution in coastal marsh fish populations. \textit{Environmental Research Laboratory, US Army Corp of Engineers}, Vicksburg, MD.
	\end{itemize}

\section*{Professional Service}
  \subsection*{Advising/Mentoring}
   \begin{itemize}
    \item H. Oliver (University of Louisiana-Lafayette, undergraduate)
	  \item J. Fisher (North Dakota State University, graduate, Ph.D.)
	  \item B. Kowalski (North Dakota State University, graduate, M.S.)
   \end{itemize}
   
  \subsection*{Peer Review}
		\begin{itemize}
  		\item Molecular Ecology, Conservation Genetics, Evolutionary Applications, \\ 
      Canadian J. of Fisheries \& Aquatic Sci., Marine Coastal Fisheries, \\ 
      Marine Ecology Progress Series, Ecology, Conservation Biology
		\end{itemize}

\section*{Miscellaneous}
  \begin{itemize}
    \item Vessel operator certification
    \item Airboat operation certification
    \item SCUBA diving (Advanced OW, Nitrox)
  \end{itemize}

  \subsection*{Computer Skills}
  	\begin{itemize}
  		\item \textbf{Languages}: R, SAS,  Python, ADMB, Perl, HTML, Javascript, Bash
  		\item \textbf{Applications}: \LaTeX, ArcGIS, GRASS, Git, Make, common Microsoft products
      \item \textbf{Operating Systems}: Unix/Linux, Windows, OS X
  	\end{itemize}
    

% \section*{References}  
% 
%   \begin{minipage}[t]{0.5\textwidth}
%     \begin{itemize}
%       \item J. Kevin Craig \\ Fisheries Biologist \\ National Marine Fishery Service \\
%         Beaufort Lab \\ Beaufort, NC 28516 \\
%         \href{mailto:kevin.craig@noaa.gov}{\tt kevin.craig@noaa.gov}
%         
%       \item Paul L. Leberg \\ Professor \\ University of Louisiana-Lafayette \\
%         Department of Biology \\ PO 42451 \\ Lafayette, LA 70504 \\
%         (337) 482-6637 \\
%         \href{mailto:leberg@louisiana.edu}{\tt leberg@louisiana.edu}
%     \end{itemize}
%   \end{minipage}
% 
%   \begin{minipage}[t]{0.5\textwidth}
%     \begin{itemize}
%       \item Joe E. Neigel \\ Professor \\ University of Louisiana-Lafayette \\
%         Department of Biology \\ PO 42451 \\ Lafayette, LA 70504 \\
%         (337) 482-5661 \\
%         \href{mailto:jneigel@louisiana.edu}{\tt jneigel@louisiana.edu}
%       \item Paul L. Klerks \\ Associate Professor \\ University of Louisiana-Lafayette \\
%         Department of Biology \\ PO 42451 \\ Lafayette, LA 70504 \\
%         (337) 482-6356 \\
%         \href{mailto:klerks@louisiana.edu}{\tt klerks@louisiana.edu}
%     \end{itemize}
%   \end{minipage}  


\bigskip

% Footer
\begin{center}
\begin{footnotesize}
Last updated: \today \\
\href{http://kevin-purcell.com/}{\tt www.kevin-purcell.com}
\end{footnotesize}
\end{center}
\end{document}
